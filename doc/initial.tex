\documentclass{article}
\usepackage[polish]{babel}
\usepackage[utf8]{inputenc}
\usepackage[T1]{fontenc}
\let\lll\undefined
\usepackage{amssymb}
\usepackage{amsmath}
\setcounter{tocdepth}{3}
\usepackage{graphicx}
\usepackage{secdot}
\sectiondot{subsection}
\usepackage[parfill]{parskip}

\usepackage{url}
\newcommand{\keywords}[1]{\par\addvspace\baselineskip
\noindent\keywordname\enspace\ignorespaces#1}

\title{Specyfikacja projektu ALHE}
\author{Ahata Valiukevich, Łukasz Wlazły}

\begin{document}
\maketitle
\begin{abstract}
W ramach projektu należało przeanalizować temat sieci autostrad i sformułować na jego podstawie zadanie optymalizacji.
\end{abstract}

\section{Ogólny opis problemu}
\subsection{Treść zadania}
Przygotować algorytm poszukujący optymalnej sieci autostrad tworzącą siatkę połączeń pomiędzy miastami z danego zbioru - położenia miast są nam znane. Rozwiązanie powinno uwzględniać miejsca zjazdów (nie mogą one znajdować się zbyt blisko siebie) oraz pozwalać na przecinanie się autostrad nie tylko w miastach.
\subsection{Założenia wstępne}
\begin{itemize}
\item Płaszczyzna - to przestrzeń $\mathbb{Z}^{2}$.
\item Miasto - punkt na płaszczyźnie. Liczba miast $M$ oraz ich współrzędne są znane i niezmienne.
\item Zjazd - najkrótszy odcinek łączący miasto z najbliższym fragmentem autostrady.
\item Autostrada - łamana opisana za pomocą $M$ punktów. Rozwiązania, które nie zapewniają połączenia wszystkich miast ze sobą, zostaną odrzucone.
\end{itemize}

\section{Przestrzeń przeszukiwań i sąsiedztwo}
Przestrzenią przeszukiwań są $M$-elementowe wektory krotek postaci $[x, y, c]$, gdzie $(x, y)$ - to współrzędne wierzchołka łamanej, a $c$ - zmienna binarna, mówiąca, czy istenieje połączenie pomiędzy poprzednim a aktualnym wierzchołkiem. Wektor krotek będzie rozważany jako lista cykliczna, tj. wartość $c$ w pierwszej krotce mówi, czy istnieje połączenie z punktem opisanym w ostatniej krotce.\\
Sąsiadami $k$-tego stopnia w zadanej przestrzeni przeszukiwań będziemy nazywać dwa wektory różniące się $k$ krotkami.

\section{Funkcja celu}
\subsection{Sposób obliczania}
Na wartość funkcji celu wpływ będą miały następujące parametry:
\begin{itemize}
\item położenie miast
\item położenie autostrad
\item koszt jednego kilometra autostrady;
\item koszt jednego kilometra zjazdu.
\end{itemize}
Koszt autostrady będzie opisany za pomocą funkcji liniowej, natomiast koszt każdego zjazdu zostanie wyrażony w postaci funkcji oferującej znacznie wyższy wzrost - wielomianowej lub wykładniczej - co zapewni występowanie krótkich i nieczęsto występujących zjazdów. Wartość funkcji celu będzie sumą kosztu autostrad i zjazdów. Optymalizacja będzie polegała na minimalizacji sumarycznego kosztu sieci autostrad.

\subsection{Wzór funkcji celu}
Niech $i$ będzie indeksem krotki w wektorze, $k$ będzie numerem miasta, a $p_k$ i $r_k$ współrzędnymi $k$-tego miasta. Załóżmy też, że wartości logiczne \texttt{true} oraz \texttt{false} odpowiadają wartościom $1$ oraz $0$. Funkcję celu definiujemy wówczas jako:
\begin{align*}
\sum_{i = 0}^{M} c_i((x_i - x_{i-1})^2 + (y_i - y_{i-1})^2) + \sum_{k = 1}^{M} 2^{f(p_k, r_k)} - 1
\end{align*}
gdzie $f(p_k, r_k)$ jest funkcją obliczającą odległość miasta od autostrady.

\subsection{Przykład}
Niech $M$ będzie liczbą miast, $C$ zbiorem punktów z przestrzeni $\mathbb{Z}^{2}$ opisujących położenie miast, a $A$ wektorem krotek opisującym budowę autostrady. Załóżmy także, że $f(z)$ jest funkcją kosztu autostrady w zależności od jej długości, a $g(w)$ funkcją kosztu zjazdu zależną od jego długości. Przez $k(A, C)$ oznaczmy sumaryczną funkcję kosztu.\\

Przyjmijmy następujące wartości:
\begin{flalign*}
& M = 3 &\\
& C = \{(2, 2), (4, 8), (6, 6)\} &\\
& A = \big[ (1, 4, F), (4, 4, T), (6, 4, T), (8, 4, T) \big] &\\
& f(z) = z &\\
& g(w) = 2^w - 1 &\\
\end{flalign*}

Łamana ma w tym przypadku długość $7$. \\
Mamy także trzy zjazdy, prowadzące od każdego z miast, odpowiednio o długości: $2$, $4$ oraz $2$.
Mając powyższe dane możemy obliczyć wartość funkcji celu:
\begin{flalign*}
& f(z) = f(7) = 7 & \\
& g(w_1) = g(w_3) = 2^2 - 1 = 3 & \\
& g(w_2) = 2^4 - 1 = 15 & \\
& k(A, C) = 7 + 3 + 15 + 3 = 28
\end{flalign*}

\section{Metody optymalizacji}
Planowane metody heurystyczne to przeszukiwanie ze zmiennym sąsiedztwem (VNS) oraz symulowane wyżarzanie.

\section{Sposób przeprowadzania eksperymentów}
\begin{enumerate}
\item Początkowe testy zostaną przeprowadzone na relatywnie małych danych, co będzie umożliwiało porównanie z wynikami obliczonymi ręcznie oraz znalezienie ewentualnych błędów.
\item Wyniki będą prezentowane na dwa sposoby: w formie tekstowej oraz graficznej.
\item Testy sprawdzające wydajność metod optymalizacji zostaną przeprowadzone odpowiednio dla $M$ równego 10, 25 oraz 50.
\end{enumerate}
\end{document}
